\documentclass[11pt,a4paper,roman]{moderncv}        % possible options include font size ('10pt', '11pt' and '12pt'), paper size ('a4paper', 'letterpaper', 'a5paper', 'legalpaper', 'executivepaper' and 'landscape') and font family ('sans' and 'roman')


\moderncvstyle{classic}                            % style options are 'casual' (default), 'classic', 'oldstyle' and 'banking'
\moderncvcolor{green}                              % color options 'blue' (default), 'orange', 'green', 'red', 'purple', 'grey' and 'black'
\renewcommand{\familydefault}{\sfdefault}         % to set the default font; use '\sfdefault' for the default sans serif font, '\rmdefault' for the default roman one, or any tex font name
\nopagenumbers{}                                  % uncomment to suppress automatic page numbering for CVs longer than one page

% character encoding
\usepackage[utf8]{inputenc}                       % if you are not using xelatex ou lualatex, replace by the encoding you are using
\usepackage{CJKutf8}                              % if you need to use CJK to typeset your resume in Chinese, Japanese or Korean

\usepackage[hidelinks]{hyperref}
\newcommand{\link}[2]{\hyperref[#1]{\color{blue}\setulcolor{green}\ul{#2}}}

% adjust the page margins
\usepackage[scale=0.75]{geometry}

% personal data
\name{Hendry}{Khoza}
\title{Application for MICT Intership}                               % optional, remove / comment the line if not wanted
\address{26}{Frangipani St}{Pretoria}% optional, remove / comment the line if not wanted; the "postcode city" and and "country" arguments can be omitted or provided empty
\phone[mobile]{+27 81-249-0306}                   % optional, remove / comment the line if not wanted
\email{h3khoza@gmail.com}                               % optional, remove / comment the line if not wanted
\homepage{www.hendry.xyz}                         % optional, remove / comment the line if not wanted

\begin{document}
%-----       letter       ---------------------------------------------------------
% recipient data
\recipient{}{Departamento, Empresa}
\date{\today}
\opening{Estimado Destinatario,}
\closing{Muchas gracias por su tiempo e interés y reciba un cordial saludo.}
% \enclosure[Adjunto]{CV}          % use an optional argument to use a string other than "Enclosure", or redefine \enclname

\makelettertitle

My name is Hendry and I'm currently doing my final year in computer engineering
at Tshwane University of Technology(TUT). I'm so excited to apply for the
Vodacom Discover Graduate program.


I've been using Vodacom ever since my first phone and I really enjoy how
reliable it is even in rural areas. My favorite service from Vodacom was
Vodacom Millionaires. I used to play it every week,  every Vodacom customer
used to get two free entries to play.  I've won airtime multiple times from
Vodacom Millionaire.

\begin{itemize} \itemsep1em 
	\item Any first-year student in Computer Engineering at TUT is required to complete 14 modules ( 7 per semester). Although it wasn't easy, I had to manage my time to be more flexible and be dedicated to my school work. That paid off, I managed to pass all the modules with an average mark of 70%.
  
	\item Due to my good pass rate mark, I was offered a position to be a Student Assistant for Technical Programming and Engineering Mathematics. One of my responsibilities as a student Assistant was to explain certain topics to students that have difficulties understanding during lecture time. This improves my communication skills and public speaking. It also helped me understand the content more because I couldn't just study to forget I had to explain what I studied in simple terms that anyone can understand.

	\item I remember when I wrote my first Hello world program It felt like I was part of the minority of people who can code in this world. My goal as a programmer was to create games. I've never got to reach my goal but trying to pursue this goal created a passion in me about writing computer programs. I've contributed to open-source projects and built my own portfolio website.

	\item Since high school, I managed to pass complicated subjects with flying colors, such as Maths, Physics, and Life Science. This carry-on even at university from Technical Programming to Engineering Mathematics. The most challenging Module was Maths 3.

	\item While I'm here at university, I make extra income by developing websites for clients. Most of my customers are small business owners who want to have their first website. A recent website that I   develop is reserved (link). Developing something that you want is much easier than developing something that a customer wants. I even spent less time developing websites and more time writing reports and Sending invoices. I'm happy about what I'm doing, it teaches me a lot to be customer-obsessed.
\end{itemize}



% \vspace{0.5cm}


\makeletterclosing

\end{document}

